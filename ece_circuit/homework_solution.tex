\documentclass[10pt]{article}
\title{ECE205 Assignment 4}
\nonstopmode
%\usepackage[utf-8]{inputenc}
\usepackage{graphicx} % Required for including pictures
\usepackage[figurename=Figure]{caption}
\usepackage{float}    % For tables and other floats
\usepackage{verbatim} % For comments and other
\usepackage{amsmath}  % For math
\usepackage{amssymb}  % For more math
\usepackage{fullpage} % Set margins and place page numbers at bottom center
\usepackage{paralist} % paragraph spacing
\usepackage{listings} % For source code
\usepackage{subfig}   % For subfigures
\usepackage{enumitem} % useful for itemization
\usepackage{siunitx}  % standardization of si units

\usepackage{tikz,bm} % Useful for drawing plots
%\usepackage{tikz-3dplot}
\usepackage[RPvoltages]{circuitikz}

%%% Colours used in field vectors and propagation direction
\definecolor{mycolor}{rgb}{1,0.2,0.3}
\definecolor{brightgreen}{rgb}{0.4, 1.0, 0.0}
\definecolor{britishracinggreen}{rgb}{0.0, 0.26, 0.15}
\definecolor{cadmiumgreen}{rgb}{0.0, 0.42, 0.24}
\definecolor{ceruleanblue}{rgb}{0.16, 0.32, 0.75}
\definecolor{darkelectricblue}{rgb}{0.33, 0.41, 0.47}
\definecolor{darkpowderblue}{rgb}{0.0, 0.2, 0.6}
\definecolor{darktangerine}{rgb}{1.0, 0.66, 0.07}
\definecolor{emerald}{rgb}{0.31, 0.78, 0.47}
\definecolor{palatinatepurple}{rgb}{0.41, 0.16, 0.38}
\definecolor{pastelviolet}{rgb}{0.8, 0.6, 0.79}
\begin{document}

\begin{center}
	\hrule
	\vspace{.4cm}
	{\textbf { \large ECE 205 --- Signals and Systems}}
\end{center}

{\textbf{Name:}\ Benjamin Feaster \hspace{\fill} \textbf{Due Date:} August 3, 2020, 9:00 AM \par
{ \textbf{Mailbox Number:}} \ 939 \hspace{60 mm} \textbf{Assignment:} 3 \vspace{.4cm} \par % /par required to remove underfull /hbox error
	\hrule 

\paragraph*{Problem 1} %\hfill \newline
 Given the following second order circuit:

\begin{figure}[h]
\centering
\begin{circuitikz}
	\draw (0,0) 
	to[american voltage source,v=$V_1$] (0,2) % The voltage source
    to[short] (2,2)
    to[R=$R_1$] (2,0) % The resistor
    to[short] (0,0);
    \draw (2,2)
    to[short] (4,2)
    to[L=$L_1$] (4,0)
    to[short] (2,0);
    \draw (4,2)
    to[short] (6,2)
    to[C=$C_1$] (6,0)
    to[short] (4,0);
 \begin{scope}[overlay]
 \end{scope} 
\end{circuitikz}

\begin{circuitikz}[scale=1]\draw
(5,.5) node [op amp] (opamp) {}
(0,0) node [left] {$U_{we}$} to [R, l=$R_d$, o-*] (2,0)
to [R, l=$R_d$, *-*] (opamp.+)
to [C, l_=$C_{d2}$, *-] ($(opamp.+)+(0,-2)$) node [ground] {}
(opamp.out) |- (3.5,2) to [C, l_=$C_{d1}$, *-] (2,2) to [short] (2,0)
(opamp.-) -| (3.5,2)
(opamp.out) to [short, *-o] (7,.5) node [right] {$U_{wy}$}
;\end{circuitikz}

\caption{}
\end{figure}

The following questions refer to Figure 1 above:

\begin{enumerate}[label=(\alph*)]
\item Find $V_1$ in terms of $R_1$, $L_1$, and $C_1$.

\begin{equation}
	V_{p1} = \left( \frac{R_1}{R_1 + R_2} \right) V_1 
\end{equation}


\end{enumerate}

\end{document}
