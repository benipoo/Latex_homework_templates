\documentclass[10pt]{article}
\title{ECE205 Assignment 4}
\nonstopmode
%\usepackage[utf-8]{inputenc}
\usepackage{graphicx} % Required for including pictures
\usepackage[figurename=Figure]{caption}
\usepackage{float}    % For tables and other floats
\usepackage{verbatim} % For comments and other
\usepackage{amsmath}  % For math
\usepackage{amssymb}  % For more math
\usepackage{fullpage} % Set margins and place page numbers at bottom center
\usepackage{paralist} % paragraph spacing
\usepackage{listings} % For source code
\usepackage{subfig}   % For subfigures
\usepackage{enumitem} % useful for itemization
\usepackage{siunitx}  % standardization of si units

\usepackage{tikz,bm} % Useful for drawing plots
%\usepackage{tikz-3dplot}
\usepackage[RPvoltages]{circuitikz}

\begin{document}

\begin{center}
	\hrule
	\vspace{.4cm}
	{\textbf { \large ECE 205 --- Signals and Systems}}
\end{center}

{\textbf{Name:}\ Benjamin Feaster \hspace{\fill} \textbf{Due Date:} August 3, 2020, 9:00 AM \par
{ \textbf{Mailbox Number:}} \ 939 \hspace{60 mm} \textbf{Assignment:} 3 \vspace{.4cm} \par
	\hrule 

\paragraph*{Problem 1}
 Given the following second order circuit:

\begin{figure}[h]
\centering
\begin{circuitikz}
	\draw (0,0) 
	to[american voltage source,v=$V_1$] (0,2)
    to[short] (2,2)
    to[R=$R_1$] (2,0)
    to[short] (0,0);
    \draw (2,2)
    to[short] (4,2)
    to[L=$L_1$] (4,0)
    to[short] (2,0);
    \draw (4,2)
    to[short] (6,2)
    to[C=$C_1$] (6,0)
    to[short] (4,0);
\end{circuitikz}

\caption{}
\end{figure}

\begin{circuitikz}[scale=1]
	\draw (5,.5) % create starting point to place objects
	node [op amp] (opamp) {} (0,0) % op amp at predefined origin
	node [left] {$V_{in}$} % create open node on left
	to [R, l=$R_a$, o-*] (2,0) % resistor at coords with open circle on left and solid cirle on right
	to [R, l=$R_b$, *-*] (opamp.+) % connect resistors and connect to the op amp + side
	to [C, l_=$C_{d}$, *-] ($(opamp.+)+(0,-2)$) % connect op amp + side to capacitor at given coords
	node [ground] {} % connect ground at the end
	(opamp.out) |- (3.5,2) % draw op amp output line path to top of op amp at given coords
	to [C, l_=$C_{e}$, *-] (2,2) % draw line from previous coords to new capacitor at coords
	to [short] (2,0) % draw line from previous capacitor to the line below at coords
	(opamp.-) -| (3.5,2) (opamp.out) % draw the - side of op amp to the node at coords to the op amp output
	to [short, *-o] (7,.5) % draw line from op amp output to coords
	node [right] {$V_{out}$}; % add a node on right side with open circle
\end{circuitikz}

The following questions refer to Figure 1 above:

\begin{enumerate}[label=(\alph*)]
\item Find $V_1$ in terms of $R_1$, $L_1$, and $C_1$.

\begin{equation}
	V_{p1} = \left( \frac{R_1}{R_1 + R_2} \right) V_1 
\end{equation}


\end{enumerate}

\end{document}
