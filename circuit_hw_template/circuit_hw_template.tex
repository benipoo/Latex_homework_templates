\documentclass[10pt]{article}

\usepackage[leqno]{amsmath}  % For maths
\usepackage{amssymb}  % For more maths
\usepackage{geometry} % For customizing margins
\usepackage[RPvoltages]{circuitikz} % For creating circuit diagrams
\usepackage{collectbox} % For creating bounding boxes
\usepackage{pagecolor,lipsum} % For setting a background color
\usepackage{siunitx} % For circuit component values

\geometry{paperwidth=8.5in, % Adjust margins and page size
paperheight=11in,
tmargin = 0.1in,
lmargin = 0.3in,
rmargin = 0.3in,}

\newcommand{\bbox}{ % made /bbox command to make bounding box around text
\collectbox{\setlength{\fboxsep}{5pt} \fbox{\BOXCONTENT}}} 

\newcommand{\s}{\vspace{0.3cm}} % made /s command to create a vertical gap between text

\begin{document}

%\pagecolor{gray!50!black} % For darker background colors to reduce white light

\begin{center} \hrule \s {\textbf { \large ECE 333 --- Latex Homework Examples}} \end{center}

{\textbf{Name:}\ John Doe \hspace{\fill} \textbf{Due Date:} August 32, 2020, 9:00 AM \par
{\textbf{Mailbox Number:}} \ 444 \hspace{93.5 mm} \textbf{Assignment:} 3 \vspace{.4cm} \par \hrule \vspace{.4cm}

\textbf{Problem 1:} Solve for $V_{out}$ in the circuit below: \s

\begin{minipage}{0.28\textwidth} 
    
	$ V_{p1} = \left(\frac{R_1}{R_1 + R_2}\right) V_1 $ \s

	$ \frac{V_{p1}-0}{R_{3}}+\frac{V_{p1}-V_{out1}}{R_{4}}=0 $ \s

	$ \frac{1}{R_{3}}V_{p1}+\frac{1}{R_{4}}V_{p1}-\frac{1}{R_{4}}V_{out1}=0 $ \s

	$ \frac{1}{R_{4}}V_{out1}=\frac{1}{R_{3}}V_{p1}+\frac{1}{R_{4}}V_{out1} $ \s

	$ V_{out1}=\left(\frac{R_{4}}{R_{3}}+1\right)V_{p1} $ \s

	$ \frac{V_{out1}}{V_{in}}=\left(\frac{R_{4}}{R_{3}}+1\right)\left(\frac{R_1}{R_1 + R_2}\right) $ \vspace{0.1cm} \s
\end{minipage}
\begin{minipage}{0.2\textwidth}

	$ \frac{0-V_{out1}}{R_5}+\frac{0-V_{out2}}{R_6}=0$ \s

	$ -\frac{V_{out1}}{R_5}-\frac{V_{out2}}{R_6}=0$ \s

	$ -\frac{V_{out1}}{R_5}=\frac{V_{out2}}{R_6}$ \s

	$ V_{out2}=-\left(\frac{R_6}{R_5}\right)V_{out1}$ 
	
	\bbox{$ \frac{V_{out}}{V_{in}}=\left(\frac{-R_6}{R_5}\right)\left(\frac{R_{4}}{R_{3}}+1\right)\left(\frac{R_1}{R_1 + R_2}\right)$} 

\end{minipage}
\hfill
\begin{minipage}{0.4\textwidth}
\begin{tabular}{p{\textwidth}}
	\begin{circuitikz}[scale=0.9]
	\draw (5,.5) % create starting point to place objects
	node [op amp] (opamp) {} (0,0) % op amp at predefined origin
	node [left] {$V_{in}$} % create open node on left
	to [R, l=$R_a$, o-*] (2,0) % resistor at coords with open circle on left and solid cirle on right
	to [R, l=$R_b$, *-*] (opamp.+) % connect resistors and connect to the op amp + side
	to [C, l_=$C_{d}$, *-] ($(opamp.+)+(0,-2)$) % connect op amp + side to capacitor at given coords
	node [ground] {} % connect ground at the end
	(opamp.out) |- (3.5,2) % draw op amp output line path to top of op amp at given coords
	to [C, l_=$C_{e}$, *-] (2,2) % draw line from previous coords to new capacitor at coords
	to [short] (2,0) % draw line from previous capacitor to the line below at coords
	(opamp.-) -| (3.5,2) (opamp.out) % draw the - side of op amp to the node at coords to the op amp output
	to [short, *-o] (7,.5) % draw line from op amp output to coords
	node [right] {$V_{out}$}; % add a node on right side with open circle
	\end{circuitikz} \s
\end{tabular}
\end{minipage} \vspace{0.5cm}
\hrule
\s
\textbf{Problem 2:} Solve for $V_{out}$ in the circuit below: \s

\begin{minipage}{0.28\textwidth} 
    
	$ V_{p1} = \left(\frac{R_1}{R_1 + R_2}\right) V_1 $ \s

	$ \frac{V_{p1}-0}{R_{3}}+\frac{V_{p1}-V_{out1}}{R_{4}}=0 $ \s

	$ \frac{1}{R_{3}}V_{p1}+\frac{1}{R_{4}}V_{p1}-\frac{1}{R_{4}}V_{out1}=0 $ \s

	$ \frac{1}{R_{4}}V_{out1}=\frac{1}{R_{3}}V_{p1}+\frac{1}{R_{4}}V_{out1} $ \s

	$ V_{out1}=\left(\frac{R_{4}}{R_{3}}+1\right)V_{p1} $ \s

	$ \frac{V_{out1}}{V_{in}}=\left(\frac{R_{4}}{R_{3}}+1\right)\left(\frac{R_1}{R_1 + R_2}\right) $ \vspace{0.1cm} \s
\end{minipage}
\begin{minipage}{0.2\textwidth}

	$ \frac{0-V_{out1}}{R_5}+\frac{0-V_{out2}}{R_6}=0$ \s

	$ -\frac{V_{out1}}{R_5}-\frac{V_{out2}}{R_6}=0$ \s

	$ -\frac{V_{out1}}{R_5}=\frac{V_{out2}}{R_6}$ \s

	$ V_{out2}=-\left(\frac{R_6}{R_5}\right)V_{out1}$ 
	
	\bbox{$ \frac{V_{out}}{V_{in}}=\left(\frac{-R_6}{R_5}\right)\left(\frac{R_{4}}{R_{3}}+1\right)\left(\frac{R_1}{R_1 + R_2}\right)$} 

\end{minipage}
\begin{minipage}{0.4\textwidth}
\begin{tabular}{p{\textwidth}} \hspace{1cm}
$L_1$ = 50\si{\micro}\si{\henry}, $R_1$ = 500\si{\micro}\si{\ohm}, $C_1$ = 700\si{\micro}\si{\coulomb} \s

	\begin{circuitikz}[scale=1.1]
	\draw (0,0) 
	to[american voltage source,v=$V_1$] (0,2)
    to[short] (2,2)
    to[R=$R_1$] (2,0)
    to[short] (0,0);
    \draw (2,2)
    to[short] (4,2)
    to[L=$L_1$] (4,0)
    to[short] (2,0);
    \draw (4,2)
    to[short] (6,2)
    to[C=$C_1$] (6,0)
    to[short] (4,0);
	\end{circuitikz} \s \s \s
	
\end{tabular}
\end{minipage} \vspace{0.5cm}
\hrule \s

\textbf{Problem 3:} Solve for $V_{out}$ in the circuit below: \s

\begin{minipage}{0.28\textwidth}
    
	$ V_{p1} = \left(\frac{R_1}{R_1 + R_2}\right) V_1 $ \s

	$ \frac{V_{p1}-0}{R_{3}}+\frac{V_{p1}-V_{out1}}{R_{4}}=0 $ \s

	$ \frac{1}{R_{3}}V_{p1}+\frac{1}{R_{4}}V_{p1}-\frac{1}{R_{4}}V_{out1}=0 $ \s

	$ \frac{1}{R_{4}}V_{out1}=\frac{1}{R_{3}}V_{p1}+\frac{1}{R_{4}}V_{out1} $ \s

	$ V_{out1}=\left(\frac{R_{4}}{R_{3}}+1\right)V_{p1} $ \s

	$ \frac{V_{out1}}{V_{in}}=\left(\frac{R_{4}}{R_{3}}+1\right)\left(\frac{R_1}{R_1 + R_2}\right) $ \vspace{0.1cm} \s
\end{minipage}
\begin{minipage}{0.2\textwidth}

	$ \frac{0-V_{out1}}{R_5}+\frac{0-V_{out2}}{R_6}=0$ \s

	$ -\frac{V_{out1}}{R_5}-\frac{V_{out2}}{R_6}=0$ \s

	$ -\frac{V_{out1}}{R_5}=\frac{V_{out2}}{R_6}$ \s

	$ V_{out2}=-\left(\frac{R_6}{R_5}\right)V_{out1}$ 
	
	\bbox{$ \frac{V_{out}}{V_{in}}=\left(\frac{-R_6}{R_5}\right)\left(\frac{R_{4}}{R_{3}}+1\right)\left(\frac{R_1}{R_1 + R_2}\right)$} 

\end{minipage}
\begin{minipage}{0.4\textwidth}
\begin{tabular}{p{\textwidth}} \hspace{1cm}
$L_1$ = 50\si{\micro}\si{\henry}, $R_1$ = 500\si{\micro}\si{\ohm}, $C_1$ = 700\si{\micro}\si{\coulomb} \s

	\begin{circuitikz}[scale=1.1]
	\draw (0,0) 
	to[american voltage source,v=$V_1$] (0,2)
    to[short] (2,2)
    to[R=$R_1$] (2,0)
    to[short] (0,0);
    \draw (2,2)
    to[short] (4,2)
    to[L=$L_1$] (4,0)
    to[short] (2,0);
    \draw (4,2)
    to[short] (6,2)
    to[C=$C_1$] (6,0)
    to[short] (4,0);
	\end{circuitikz} \s \s \s
	
\end{tabular}
\end{minipage} \vspace{0.5cm}
\hrule \s

\end{document}





